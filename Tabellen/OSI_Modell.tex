\renewcommand{\tabcolsep}{1pt}
\begin{longtable}{@{}p{0.3\textwidth}@{\hspace{3em}}p{0.6\textwidth}}
    \\\hline
    \makecell[l]{Bitübertragungsschicht \\ (Physical Layer)}
        & 
        \begin{itemize}
            \item elektrische / physische Übertragung der Daten 
        \end{itemize}
        
    \\\hline

    \makecell[l]{Sicherungsschicht \\ (Data Link Layer)}
        &
        \begin{itemize}
            \item alle Vorkehrungen, die dafür sorgen, dass aus der physikalischen Übertragung ein verlässlicher Datenfluss wird
        
        \end{itemize}
    \\\hline
    
    \makecell[l]{Vermittlungsschicht \\ (Network Layer)}
        &
        \begin{itemize}
            \item Komponenten und Protokolle, die an der Verbindung zwischen Rechnern beteiligt sind - das sogenannte Routing - Weiterleiten von Daten in andere logische und oder physikalisch inkompatible Netzwerke
        \end{itemize}
    \\\hline
    
    \makecell[l]{Transportschicht \\ (Transport Layer)}
        &
        \begin{itemize}
            \item verbindungsorientierte Protokolle wie TCP und verbindungslose Protokolle wie UDP
            \item ein wichtiger Aspekt dieser Schicht ist Multiplexing - Anbindung der Datenpakete an konkrete Prozesse auf den kommunizierenden Rechnern 
            \item Segmentierung des Datenstroms und Datenstauvermeidung
        \end{itemize}
    \\\hline

    \makecell[l]{Kommunikationssteuerungsschicht \\ (Session Layer)}
        &
        \begin{itemize}
            \item sichert Kommunikation zwischen kooperierenden Anwendungen oder Prozessen auf verschiedenen Rechnern
            \item organisiert und synchronisiert Datenaustausch
        \end{itemize}
    \\\hline

    \makecell[l]{Darstellungsschicht \\ (Presentation Layer)}
        &
        \begin{itemize}
            \item Konvertierung und Übertragung von Datenformaten, Datensätzen, Zeichensätzen, grafische Anweisungen und Dateidienste
            \item systemabhängige Darstellung von Daten
            \item Datenkompression, Verschlüsselung
            \item stellt sicher, dass Daten die von der Anwendungsschicht des einen Systems gesendet werden von der Anwendungsschicht eines anderen Systems gelesen werden können
        \end{itemize}
    \\\hline

    \makecell[l]{Anwendungsschicht \\ (Application Layer)}
        &
        \begin{itemize}
            \item unmittelbare Kommunikation zwischen Benutzeroberflächen der Anwendungsprogramme
            \item Das Anwendungsprogramm selbst zählt nicht dazu
        \end{itemize}
    \\\hline

    
\end{longtable}
